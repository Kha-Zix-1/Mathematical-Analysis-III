%!TEX TS-program = xelatex
\documentclass[xetex]{beamer}

\usefonttheme{professionalfonts}

\usepackage[UTF8]{ctex}
\usepackage{hyperref}
\usepackage{unicode-math}
\usepackage{amsmath, amssymb}
\usepackage{graphicx, wrapfig}

\usepackage{nopageno}

\graphicspath{{./img/}}

\usetheme[block=fill, subsectionpage=progressbar]{metropolis}

\setmathfont{XITS Math}

\begin{document}
	\begin{frame}
		\title{各种积分间的联系与场论初步}
		\subtitle{各种积分间的联系}
		\author{数学分析MOOC小组}
		\date{  }
		\titlepage
	\end{frame}
	
	\begin{frame}
		\frametitle{概览}
		\begin{itemize}
			\item[1.] 知识回顾
				\begin{itemize}
					\item[1.1] 格林公式
					\item[1.2] 高斯公式
					\item[1.3] 斯托克斯公式
				\end{itemize}			
			\item[2.] 习题讲解
		\end{itemize}
	\end{frame}
	
	\begin{frame}
		\section{知识回顾}
	\end{frame}
	
	\begin{frame}
		\frametitle{格林公式}
		\textbf{定理22.1}
			设$D$是由逐段光滑闭曲线$L$围成的平面单连通闭区域,函数$P(x, y)$,$Q(x, y)$在$D$上有一阶连续偏导数,则有
			\begin{equation*}
				\iint_D (\frac{ \partial Q }{ \partial x } - \frac{ \partial P }{ \partial y }) \mathrm{d} x \mathrm{d} y = \oint_L P\mathrm{d} x + Q\mathrm{d} x
			\end{equation*}
			其中右端的积分路径是闭曲线$L$,方向取正向。
	\end{frame}
	
	\begin{frame}
		\frametitle{高斯公式}
		\textbf{定理22.2}
		设空间区域$V$由分片光滑的双侧封闭曲面$S$围成,函数$P(x, y, z)$,$Q(x, y, z)$,$R(x, y, z)$在$V$及$S$上有一阶连续偏导数,则有
		\begin{equation*}
			\iiint_V (\frac{ \partial P }{ \partial x } + \frac{ \partial Q }{ \partial y } + \frac{ \partial R }{ \partial z })\mathrm{d} x \mathrm{d} y \mathrm{d} z = \oiint _S P\mathrm{d} y \mathrm{d} z + Q\mathrm{d} z \mathrm{d} x + R\mathrm{d} x \mathrm{d} y
		\end{equation*}
		其中$S$的方向为外侧。
	\end{frame}
	
	\begin{frame}
		\frametitle{斯托克斯公式}
		\textbf{定理22.3}
		若光滑曲面$S$的边界为光滑曲线$L$,函数$P$、$Q$、$R$在曲面$S$及曲线$L$上具有连续的一阶偏导数,则
		\begin{equation*}
			\begin{split}
			&\iint_S(\frac{\partial R}{\partial y} - \frac{\partial Q}{\partial z})\mathrm{d} y \mathrm{d} z + 
			(\frac{\partial P}{\partial z} - \frac{\partial R}{\partial x})\mathrm{d} z \mathrm{d} x +
			(\frac{\partial Q}{\partial x} - \frac{\partial P}{\partial y})\mathrm{d} x \mathrm{d} y \\
			= & \iint_S
				\begin{vmatrix} 
					\mathrm{d} y \mathrm{d} z & \mathrm{d} z \mathrm{d} x & \mathrm{d} x \mathrm{d} y \\ 
					\frac{\partial }{\partial x} & \frac{\partial }{\partial y} & \frac{\partial }{\partial z} \\
					P & Q & R\\
				\end{vmatrix}\quad\\
			= & \oint_L P\mathrm{d} x + Q\mathrm{d} y + R\mathrm{d} z\\
			\end{split}
		\end{equation*}
	\end{frame}
	
	
	\begin{frame}
		\section{习题讲解}
	\end{frame}
	
	\begin{frame}
		\frametitle{例2}
		计算$I = \oint_L \frac{x \mathrm{d}y - y\mathrm{d}x}{x^2 + y^2}$,其中$L$为$x^2 + y^2 = a^2$的正向。
		
		\textbf{解:}
			在$L$上有$x^2 + y^2 = a^2$,故有
			\begin{equation*}
				I = \frac{1}{a^2}\oint_L x \mathrm{d}y - y\mathrm{d}x
			\end{equation*}
			取$P(x, y) = -y$,$Q(x, y) = x$,由格林公式得
			\begin{equation*}
				I = \frac{1}{a^2}\iint_D 2 \mathrm{d}x \mathrm{d}y = \frac{2}{a^2}\pi a^2 = 2\pi
			\end{equation*}
	\end{frame}
	\begin{frame}
		\frametitle{习题3(2)}
		$\iint_S x^3 \mathrm{d} y \mathrm{d} z + y^3 \mathrm{d} z \mathrm{d} x + z^3 \mathrm{d} x \mathrm{d} y$,其中$S$是单位球面得外侧。
		
		\textbf{解:}
			由题意,知道$x^2 + y^2 + z^2 = 1$,设函数$P(x, y, z) = x^3$,$Q(x, y, z) = y^3$,$R(x, y, z) = z^3$,则由高斯公式得
			\begin{equation*}
				\begin{split}
				& \iint_S x^3 \mathrm{d} y \mathrm{d} z + y^3 \mathrm{d} z \mathrm{d} x + z^3 \mathrm{d} x \mathrm{d} y \\
		      = & \iiint_V (3x^2 + 3y^2 + 3z^2) \mathrm{d} x \mathrm{d} y \mathrm{d} z \\
			  = & 3 \iiint_V  \mathrm{d} x \mathrm{d} y \mathrm{d} z \\
			  = & 3 \times \frac{4\pi}{3}\\
			  = & 4\pi
				\end{split}
			\end{equation*}
	\end{frame}
	\begin{frame}
		\section{谢谢大家!}
	\end{frame}
\end{document}