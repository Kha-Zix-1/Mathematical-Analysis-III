%!TEX TS-program = xelatex
\documentclass[xetex]{beamer}

\usefonttheme{professionalfonts}

\usepackage[UTF8]{ctex}
\usepackage{hyperref}
\usepackage{unicode-math}
\usepackage{amsmath, amssymb}
\usepackage{graphicx, wrapfig}

\usepackage{nopageno}

\graphicspath{{./img/}}

\usetheme[block=fill, subsectionpage=progressbar]{metropolis}

\setmathfont{XITS Math}

\begin{document}
	\begin{frame}
		\title{重积分}
		\subtitle{重积分化累次积分}
		\author{数学分析MOOC小组}
		\date{ }
		\titlepage
	\end{frame}
	
	\begin{frame}
		\frametitle{概览}
		\begin{itemize}
			\item[1.] 知识回顾
				\begin{itemize}
					\item[1.1] 二重积分化累次积分
					\item[1.2] 三重积分化累次积分
				\end{itemize}			
			\item[2.] 习题讲解
		\end{itemize}
	\end{frame}
	
	\begin{frame}
		\section{知识回顾}
	\end{frame}
	
	\begin{frame}
		\frametitle{二重积分化累次积分}
		\textbf{定理20.1}
			
			若$f(x,y)$在矩形区域$D = [a, b] \times [c, d]$上可积,并且对$[a,b]$上的任何$x$,含参变量积分
			\begin{equation*}
				A(x) = \int_{c}^{d} f(x,y) \mathrm{d} y
			\end{equation*}
			存在,则
			\begin{equation*}
				\iint_{D} f(x, y) \mathrm{d} x \mathrm{d} y = \int_{a}^{b} \mathrm{d} x \int_{c}^{d} f(x, y) \mathrm{d} y
			\end{equation*}
		
		\textbf{推论 1}
		
			设$f(x,y)$在$D = [a, b] \times [c, d]$上连续,则
			\begin{equation*}
				\iint_{D} f(x, y) \mathrm{d} x \mathrm{d} y = \int_{a}^{b} \mathrm{d} x \int_{c}^{d} f(x, y) \mathrm{d} y =  \int_{c}^{d} \mathrm{d} y \int_{a}^{b} f(x, y) \mathrm{d} x
			\end{equation*}
	\end{frame}
	
	\begin{frame}
		\frametitle{二重积分化累次积分}
		\textbf{定理20.2}
		
			设$D = \{(x,y) | y_1(x) \leq y \leq y_2(x), a \leq x \leq b \}$,$y_1(x)$,$y_2(x)$在$[a, b]$上连续,$f(x,y)$在$D$上连续,则
			\begin{equation*}
				\iint_{D} f(x, y) \mathrm{d} x \mathrm{d} y = \int_{a}^{b} \mathrm{d} x \int_{y_1(x)}^{y_2(x)} f(x, y) \mathrm{d} y
			\end{equation*}
	\end{frame}
	
	\begin{frame}
		\frametitle{三重积分化累次积分}
		\begin{itemize}
			\item[1.] 
				\begin{equation*}
					\iiint_{V} f(x,y,z)\mathrm{d} x \mathrm{d} y\mathrm{d} z = \iint_D \mathrm{d}x \mathrm{d} y \int_{e}^{f} f(x,y,z) \mathrm{d} z
				\end{equation*}
			\item[2.]
				\begin{equation*}
					\iiint_{V} f(x,y,z)\mathrm{d} x \mathrm{d} y\mathrm{d} z = \iint_D \mathrm{d}x \mathrm{d} y \int_{z_1(x, y)}^{z_2(x, y)} f(x,y,z) \mathrm{d} z
				\end{equation*}
			\item[3.]
				\begin{equation*}
					\iiint_{V} f(x,y,z)\mathrm{d} x \mathrm{d} y\mathrm{d} z = \int_{e}^{f} \mathrm{d} z \iint_D f(x,y,z) \mathrm{d}x \mathrm{d} y 
				\end{equation*}
		\end{itemize}
	\end{frame}
	
	
	\begin{frame}
		\section{习题讲解}
	\end{frame}
	
	\begin{frame}
		\frametitle{习题1(2)}
		计算二重积分
		\begin{equation*}
			\iint_D cos(x + y) \mathrm{d}x \mathrm{d}y, D = [0, \frac{\pi}{2}] \times [0, \pi]
		\end{equation*}
		
		\textbf{解:}
			\begin{equation*}
				\begin{split}
				& \iint_D cos(x + y) \mathrm{d}x \mathrm{d}y \\
			=   & \int_{0}^{\frac{\pi}{2}} \mathrm{d}x \int_{0}^{\pi} cos(x + y) \mathrm{d} y\\ 
			=   & \int_{0}^{\frac{\pi}{2}} (\sin (x + \pi) - \sin (x)) \mathrm{d} x\\
			=   & -2\\
				\end{split}
			\end{equation*}
	\end{frame}
	\begin{frame}
		\frametitle{习题8(2)}
		计算三重积分
		\begin{equation*}
			\iiint_V z \mathrm{d}x \mathrm{d}y \mathrm{d}z
		\end{equation*}
		
		其中$V$ 由曲面 $z = x^2 + y^2$,$z = 1$,$z = 2$所围成。
		
		\textbf{解:}
		\begin{equation*}
			\begin{split}
					& \iiint_V z \mathrm{d}x \mathrm{d}y \mathrm{d}z \\
				=   & \int_{1}^{2} z \mathrm{d}z \iint_D \mathrm{d}x \mathrm{d}y \\
				=	& \int_{1}^{2} \pi z^2 \mathrm{d}z\\
				=   & \frac{7}{3} \pi \\
			\end{split}
		\end{equation*}
	\end{frame}
	
	\begin{frame}
		\section{谢谢大家!}
	\end{frame}
\end{document}