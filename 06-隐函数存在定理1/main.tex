%!TEX TS-program=xelatex
\documentclass[xetex]{beamer}
% 将上面这一行修改成下面这个样子,可以创建适合于发布的版本,这去除了所有的动画
%\documentclass[xetex, handout]{beamer}

% 规范注意:

% 使用正确的主题(beamer主题、文字字体)
% 使用正确的title信息(title、subtitle、author、date)
% 合理使用frame 和 standout frame
% 块(block、exampleblock、alertblock)以及文字段落内部的\alert的使用
% 使用图片需要使用\begin{figure}...\end{figure}并附带\caption和\label信息
% 使用enumerate和itemize组织你的点
% 使用section给你的幻灯片分部分
% 公式,文字段落内嵌公式和单独的公式块的使用
% \DeclareMathOperator的使用,以及学会在网上查找你不知道怎么输入的数学符号
% 动画的使用
% 讲课录制使用什么版本的文档;对外发布使用什么版本的文档(handout)

\usefonttheme{professionalfonts}

\usepackage[UTF8]{ctex}
\usepackage{hyperref}
\usepackage{unicode-math}
\usepackage{amsmath, amssymb}
\usepackage{graphicx, wrapfig}

\usepackage{nopageno}

\DeclareMathOperator{\argmax}{argmax}

\usetheme[block=fill]{metropolis}

\setmathfont{XITS Math}

\title{隐函数存在定理I}
\author{数学分析MOOC小组}
\date{}

\begin{document}

\frame{\maketitle}

\begin{frame}
    \frametitle{单个方程确定的隐函数}
	由方程$F(x,y)=0$,在某些条件成立的情况下,可以确定从$x$域到$y$域的映射,也就是隐函数$y=f(x)$\\
	具体的证明过程课本已经说的比较清楚了\\
	这里先为大家分析一下\\
	隐函数存在定理的各个条件是怎么得到从从$x$域到$y$域的映射的\\
	由于课本已经有了一元的情形,我们就拿多元来举例子吧\\
	对于方程$F(x_1,x_2,...,x_n,y)=0$,设点$P_0=(\dot x_1,\dot x_2,...,\dot x_n,\dot y)$\\
	函数$F$满足\\
	\begin{enumerate}
		\item
		$F$连续
		\item 
		所有的一阶偏导在点$P_0$附近的邻域$U(P_0,\delta)$内连续。
		\item
		$F(P_0)=0$
		\item
		$F_y(P_0)\not = 0$
	\end{enumerate}


\end{frame}

\begin{frame}
	\frametitle{单个方程确定的隐函数}
	参考课本一元情况的证明,我们不妨设$F_y(P_0)>0$\\ \pause
	由$F_y$的保号性\quad$\exists b>0 \quad \forall y \in [\dot y-b,\dot y+b],F_y(\dot x_1,\dot x_2,...,\dot x_n,y)>0$\\ \pause
	也就是对于固定的$\vec x=(x_1,x_2,...,x_n)$\\
	$F(x_1,x_2,...,x_n,y)$关于$y$在区间$[\dot y-b,\dot y+b]$上单调递增\\ \pause
	又因为$F(P_0)=F(\dot x_1,\dot x_2,...,\dot x_n,\dot y)=0$\\
	结合单调性可以判断$F(\dot x_1,\dot x_2,...,\dot x_n,\dot y-b)<0\quad F(\dot x_1,\dot x_2,...,\dot x_n,\dot y+b)>0$\\ \pause
	下面,我们固定$y=\dot y-b,y=\dot y+b$\\
	$g(x_1,x_2,...,x_n)=F(x_1,x_2,...,x_n,\dot y-b)$\\
	$h(x_1,x_2,...,x_n)=F(x_1,x_2,...,x_n,\dot y+b)$\\
	
\end{frame}

\begin{frame}
	\frametitle{单个方程确定的隐函数}
	由前面可以确定$g(\dot x_1,\dot x_n,...,\dot x_n)<0\quad h(\dot x_1,\dot x_n,...,\dot x_n)>0$\\
	由连续函数的保号性,存在$\delta_1>0,\delta_2>0$\\
	使得当$max\{|x_i-\dot x_i|<\delta_1\},i \in \{1,2,...,n\}$时,$g(x_1,x_2,...,x_n)<0$\\
	同理$max\{|x_i-\dot x_i|<\delta_2\},i \in \{1,2,...,n\}$时,$h(x_1,x_2,...,x_n)>0$\\ \pause
	所以取$\delta=min\{\delta_1,\delta_2 \}$\\
	当$(x_1,x_2,...,x_n)$满足$max\{|x_i-\dot x_i|<\delta\}$,有$F(x_1,x_2,...,x_n,\dot y-b)<0\quad F(x_1,x_2,...,x_n,\dot y+b)>0$\\ \pause
	由介值定理,对于每一个在方形邻域内的$(\overline{x_1},\overline{x_2},...\overline{x_n})$\\
	存在对应的$\overline{y}$\quad 使得$F(\overline{x_1},\overline{x_2},...\overline{x_n},\overline{y})=0$\\ \pause
	因此也就确定了一个隐函数$y=f(x_1,x_2,...,x_n)$\\
\end{frame}

\begin{frame}

\frametitle{单个方程确定的隐函数}
	下面我们来做一道例题\\
	P230 5.\\
	方程$xy+zlny+e^{xz}=1$在$P_0(0,1,1)$的某邻域内能否确定出一个变量是另外两个变量的函数\\ \pause
	这里由于我们不知道哪个变量是另外两个变量的函数。在直接求导时将三个变量都视为自变量,不要用链式法则\\
	设$F(x,y,z)=xy+zlny+e^{xz}-1=0$\\ \pause
	$F_x=y+ze^{xz}$,$F_y=x+\frac{z}{y}$,$F_z=lny+xe^{xz}$\\
	代入$P_0(0,1,1)$得$F_x(P_0)=2,F_y(P_0)=1,F_z(P_0)=0$\\ \pause
	且$F,F_x,F_y,F_z$在$(0,1,1)$的小邻域都连续\\
	因此由隐函数存在定理\\
	可以确定隐函数$x=x(y,z)\quad y=y(x,z)$\\ \pause
	单个方程的情形比较简单,主要是为了方程组确定隐函数做铺垫
\end{frame}

\begin{frame}[standout]
	谢谢
\end{frame}

\end{document}